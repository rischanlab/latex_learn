% This Latex Document created by %
% Rischan Mafrur %
% rischanlab@gmail.com
% rischanlab.github.io, ourmasjid.me
% May 15, 2014

\documentclass[conference]{IEEEtran}
% this is package for cite
\usepackage{cite}
% this is package for algorithm 
\usepackage{algorithm}
\usepackage{algorithmic}
% adding many math symbol
\usepackage{amsmath}
\usepackage{amssymb}

\begin{document}
\title{This is My Paper broh}
\author{
	\IEEEauthorblockN{Rischan, Mafrur}
	\IEEEauthorblockA{School of Electrical Engineering\\
	Chonnam National University\\
	Gwangju, South Korea\\
	Email : rischanlab@gmail.com}
}

\maketitle

\begin{abstract}
This is abtract of my paper
\end{abstract}

\section{Introduction}
\label{sec:intro}
In this intro we can make a references to the Methodologies section ~\ref{sec:meth} . According to \cite{contoh1}
In this section I just wanna to try using basic formating
First on is about \textbf{Bold} and then \textit{Italic} and the last is {\it Just try i dont know what is this} 

We can use also \huge{huge} \normalsize{Normalize}
This is an example of equation
$ \sqrt[n]{1+x+x^2+x^3+\ldots} $

The equation again is $ x_i = y_i^f $



\section{Methodologies}
\label{sec:meth}



Generally want the most probable hypothesis given the training data


{\huge Bayes Theorem  }

\[ P(h|D) = \frac{P(D|h) P(h)}{P(D)} \]


 $P(h)$ = prior probability of hypothesis $h$
 $P(D)$ = prior probability of training data $D$
 $P(h|D)$ = probability of $h$ given $D$
 $P(D|h)$ = probability of $D$ given $h$

{\em Maximum a posteriori} hypothesis $h_{MAP}$:
\begin{eqnarray}
& h_{MAP} & = \arg \max_{h \in H} P(h|D)\nonumber \\
& & = \arg \max_{h \in H} \frac{P(D|h) P(h)}{P(D)} \nonumber \\
& & = \arg \max_{h \in H}P(D|h) P(h) \nonumber
\end{eqnarray}

\bigskip

$ \omega  \sigma \Sigma $



\section{Result and Discussion}

%writing algorithm in latex

\section{Algorithm in LaTex}
\begin{algorithm}
	\begin{algorithmic}
		\STATE Short the input set $S_i$
		\STATE Binary Search for $x$ in $S_i$
			\FORALL {$i$ in $S_i$}
				\STATE find $x$ in $S_i$ such that $x$ = $i$
				\IF {$x$ = 0}
					\STATE { $x$ was less than the minimum threshold }
					\ENDIF
			\ENDFOR 
	\end{algorithmic}
	\caption{Algorithm for stupid work}
	\label{algo:stupid}
\end{algorithm}

\section{Math Symbol}
This is inline $n$ math symbol $\frac{n}{2}$
$
\left[
	\begin{array}{ccc}
	1&2&3\\
	1&5&6\\
	1&6&8
	\end{array}
\right]
$



\section{Conclusion}
In this section we can see in the algorithm on section \ref{algo:stupid} .


\section{References}
\bibliographystyle{IEEEtran}
\bibliography{IEEEexample}


\end{document}